%-----------------------------------------------------------------------------------------
% Determines the type of document and font size
%-----------------------------------------------------------------------------------------
\documentclass[12pt,a4paper]{article}
\usepackage[utf8]{inputenc}
\usepackage{float} % here for H placement parameter

%-----------------------------------------------------------------------------------------
% Font control
%-----------------------------------------------------------------------------------------
\usepackage{mathptmx} % Hack
%-----------------------------------------------------------------------------------------
\usepackage{helvet} % Arial/Helvetica font
\renewcommand{\familydefault}{\sfdefault} % Makes serif text all Helvetica
%-----------------------------------------------------------------------------------------
% Set up the page margins
%-----------------------------------------------------------------------------------------
\usepackage[left=2.5cm, right=2.5cm, top=2.5cm]{geometry} % Sets the page margins
%-----------------------------------------------------------------------------------------
% Allow graphics
%-----------------------------------------------------------------------------------------
\usepackage{graphicx}

%-----------------------------------------------------------------------------------------
% Add your report title here
%-----------------------------------------------------------------------------------------
\title{\huge{\textbf{Aritmetični kodirnik}}}

% Add your name here
\author{
        David Vučković \\
                Telekomunikacija 2.letnik\\
        FERI\\
        Univerza v Mariboru}
\date{\today}

%-----------------------------------------------------------------------------------------
% The start of the document
%-----------------------------------------------------------------------------------------
\begin{document}

%-----------------------------------------------------------------------------------------
% This adds the title page
%-----------------------------------------------------------------------------------------
\maketitle
\thispagestyle{empty}

\clearpage % moves to the next page

%-----------------------------------------------------------------------------------------
% This adds the abstract
%-----------------------------------------------------------------------------------------

\thispagestyle{empty}

%-----------------------------------------------------------------------------------------
% Move to a new page and set the page numbering from here
%-----------------------------------------------------------------------------------------
\clearpage % moves to the next page
\pagenumbering{arabic}

\section{Tabela kompresije} % The start of a new section

\begin{table}[H]
\begin{tabular}{llll}
    Datoteka     & Original & Kompresirana & Razmerje \\ \hline
    alice30.txt  & 148,545 bytes                & 84,819 bytes          & 0.57099868726           \\
    assassin.wav & 288,849 bytes                & 149,920 bytes         & 0.51902551159           \\
    lena.bmp     & 2,169,254 bytes              & 1,890,195 bytes             & 0.87111822066           \\
    lorem.txt    & 39,605 bytes                      & 21,405 bytes            & 0.54046206287           \\
\end{tabular}
\end{table}
\section{Tabela  kompresijskih časov}
\begin{table}[H]
    \begin{tabular}{llll}
    Datoteka     & Čas izvajanja \\ \hline
    alice30.txt  & 0.03s              \\
    assassin.wav & 0.049s                      \\
    lena.bmp     & 0.502s              \\
    lorem.txt    & 0.009                            \\
    \end{tabular}
\end{table}
\section{Tabela dekompresijskih časov pred optimizacijo}
\begin{table}[H]
    \begin{tabular}{llll}
    Datoteka     & Čas izvajanja \\ \hline
    alice30.txt  & 0.287s              \\
    assassin.wav & 0.293s                      \\
    lena.bmp     & 1.995s              \\
    lorem.txt    & 0.043                           \\
    \end{tabular}
\end{table}
\section{Tabela verjetnosti}
Tabela verjetnosti je polje velikosti 256, kjer index predstavlja simbol,
vrednost na tistem indeksu pa predstavlja število pojavitev v tekstu. Torej če imamo tekst AAB, je na polju na indeksu 64(array[64]) vrednost 2.
Vsak element je uint32.
\section{Optimizacije in pohitritve}
Pri branju iz datoteke za dekompresijo, ko pridobimo intervale, nam na klasičen način pretvarjanje intervala v simbol lahko vzame veliko časa, saj za vsak simbol iteriramo čez potencialno cel 256 elementno polje. Tukaj lahko implementiramo binarno iskanje. Kot enostavnejša optimizacija zamenjamo vse množitve z 2 s seštevanjem.
\section{Tabela optimizacije}

\begin{table}[H]
    \begin{tabular}{llll}
    Datoteka     & Optimiziran čas(t2)     & Neoptimiziran čas(t1) & Faktor pohitritve(t1/t2) \\ \hline
    alice30.txt  & 0.244s  & 0.287s        & 1.17            \\
    assassin.wav & 0.246s  & 0.293s        & 1.19s       \\
    lena.bmp     & 1.897s  & 1.995s        & 1.05s           \\
    lorem.txt    & 0.041s  & 0.043s        & 1.04s          \\
    \end{tabular}
\end{table}
\end{document}